% Created 2019-04-16 ter 14:45
% Intended LaTeX compiler: pdflatex
\documentclass[11pt]{article}
\usepackage[utf8]{inputenc}
\usepackage[T1]{fontenc}
\usepackage{graphicx}
\usepackage{grffile}
\usepackage{longtable}
\usepackage{wrapfig}
\usepackage{rotating}
\usepackage[normalem]{ulem}
\usepackage{amsmath}
\usepackage{textcomp}
\usepackage{amssymb}
\usepackage{capt-of}
\usepackage{hyperref}
\usepackage{minted}
\author{Rafael Campos Nunes}
\date{\today}
\title{A web scrapper case}
\hypersetup{
 pdfauthor={Rafael Campos Nunes},
 pdftitle={A web scrapper case},
 pdfkeywords={},
 pdfsubject={},
 pdfcreator={Emacs 26.1 (Org mode 9.1.9)},
 pdflang={English}}
\begin{document}

\maketitle
\tableofcontents


\section{Requirements}
\label{sec:org0c31e00}


The objective of the web scrapper is to list every product with the following
characteristics encoded as a JSON:

\begin{enumerate}
\item Name of the product
\item Link to one or more of its images
\item Price of the product
\end{enumerate}

The page from which all of this information will be scrapped is:
\url{https://nerdstore.com.br/categoria/especiais/game-of-throne/s}

\section{Inspection}
\label{sec:org0f548ea}

When inspecting the page I found that the elements are enclosed in the following
structure:

\begin{minted}[frame=lines,linenos=true]{html}
<ul class="products ...">
  <li class="... product ...">
    <a class="woocommerce-LoopProduct-link woocommerce-loop-product__link">
      <img class="" src="">
      <img class="" src="">
      <h2 class="woocommerce-loop-product__title">
        <span class="ellip">text ellipsed?</span>
        <span class="ellip-line">text ellipsed 2?</span>
      </h2>
      <span class="price">
        <span class="woocommerce-Price-amount amount">
          <span class="woocommerce-Price-currencySymbol">R$</span>
          "49,90"
        </span>
      </span>
    </a>
  </li>
  ...
</ul>
\end{minted}

By inspection I see that a product may or may not have more than one image and
that are specific classes to grab the specified element, though this shouldn't
be enough I think because they (the developers) might change on the long run. I
should return a JSON that looks somewhat like this:

\begin{minted}[frame=lines,linenos=true]{javascript}
o = {
  'name'  : '',
  'img'   : ['', ..., ''],
  'price' : '44,90'
}
\end{minted}

The price comes with a currency tag and that should be important to have but
it's not asked by the head hunters.

What I should do now is to take the big block of products that are contained
inside the \(ul\) tag with the \(products\) class and work from there.
\end{document}
